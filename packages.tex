% Encoding
	%\usepackage[utf8]{inputenc}
	%\usepackage[latin1]{inputenc}
	\usepackage[T1]{fontenc}

% Language
	%\usepackage[francais,english]{babel} % Second language = main language

% Show a summary of the layout of the current document with \layout.
	%\usepackage{layout}

% For easy management of document margins and the document page size.
	%\usepackage[top=2cm, bottom=1.8cm, left=1.8cm, right=1.8cm, head=14pt, foot=36pt]{geometry}

% Lets you change line spacing.
	%\usepackage{setspace}

% Euro symbol
	%\usepackage{eurosym}

% Fonts (include only one)
	%\usepackage{bookman}
	%\usepackage{charter}
	%\usepackage{newcent}
	\usepackage{lmodern}
	%\usepackage{mathpazo}
	%\usepackage{mathptmx}

% Enables typesetting of hyperlinks
	%\usepackage{url}
	\usepackage{hyperref}

% Verbatim environment
	%\usepackage{verbatim}
	%\usepackage{moreverb}
	\usepackage{fancyvrb}

% Code listing
	\usepackage{listings}

% To change header and footer of any page of the document.
	%\usepackage{fancyhdr}

% Allows you to insert graphic files within a document.
	%\usepackage{graphicx}

% Allows figures or tables to have text wrapped around them.
	%\usepackage{wrapfig}

% Adds support for colored text.
	\usepackage{xcolor}

% Allows tables rows and columns to be colored, and even individual cells.
	%\usepackage{colortbl}

% Mathematics
	\usepackage{amsmath}
	\usepackage{amssymb}
	\usepackage{mathrsfs}
	%\usepackage{asmthm}
	%\usepackage{mathtools}
	%\usepackage{bm} % Greek letters in math mode

% Provide the array and tabular environments
	\usepackage{array}

% Provide the tabularx environment
	\usepackage{tabularx}

% Provide the multirow command
	\usepackage{multirow}

% Provides control over the layout of the three basic list environments: enumerate, itemize and description.
	%\usepackage{enumitem}

% Interface to sectioning commands for selection from various title styles
	%\usepackage[nobottomtitles]{titlesec}

% Highly customized stacking of objects, insets, baseline changes, etc.
	\usepackage{stackengine}

% Routines for constrained scaling and stretching of objects, relative to a reference object or in absolute terms
	\usepackage{scalerel}

% Provides control over the typography of the Table of Contents, List of Figures and List of Tables, and the ability to create new ‘List of ...’.
	%\usepackage{tocloft}
	%\usepackage{titletoc}

% Advanced bibliography handling.
	%\usepackage{bibtex}
	%\usepackage{biblatex}

% Allows customization of appearance and placement of captions for figures, tables, etc.
	%\usepackage{caption}

% Provides the multicols environment which typesets text into multiple columns.
	%\usepackage{multicol}

% This package simplifies the insertion of external multi-page PDF or PS documents.
	%\usepackage{pdfpages}

% Prints out all index entries in the left margin of the text.
	%\usepackage{showidx}

% Allow to define multiple floats (figures, tables) within one environment giving individual captions and labels in the form 1a, 1b.
	%\usepackage{subcaption}

% Lets you insert notes of stuff to do with the syntax \todo{Add details.}.
	%\usepackage{todonotes}

% Text Companion fonts, which provide many text symbols (such as baht, bullet, copyright, musicalnote, onequarter, section, and yen), in the TS1 encoding.
	%\usepackage{textcomp}

% Floating elements placement
	\usepackage{float}

% Add document elements like a bibliography or an index to the Table of Contents.
	\usepackage[notindex,nottoc,notlot,notlof]{tocbibind}

% Allow TeX pictures or other TeX code to be compiled standalone or as part of a main document
	\usepackage{standalone}

% PGF-TikZ
	\usepackage{pgf}
	\usepackage{tikz}
