%!TEX program = xelatex
\documentclass[11pt,class=book]{standalone}
\usepackage[utf8]{inputenc}
\usepackage{fontenc}
\usepackage[table,svgnames]{xcolor}

\usepackage{pgf}
\usepackage{tikz}

\usepackage{array}
\usepackage{tabularx}
\usepackage{multirow}

\usetikzlibrary{shapes}
\usetikzlibrary{arrows.meta}
\usetikzlibrary{calc}

\definecolor{bg_color}{RGB}{250,250,229}

\colorlet{color1}{cyan!50}
\colorlet{color2}{red!30!green!40}
\colorlet{color3}{orange!50}
\colorlet{color4}{violet!60!blue!55}


\begin{document}
	\pgfmathsetmacro\computerwidth{210}%
	\pgfmathsetmacro\computerheight{200}%
	\pgfmathsetmacro\xstartmodules{15}%
	\pgfmathsetmacro\ystartmodules{-10}%
	\pgfmathsetmacro\yshiftmodules{-5}%
	\pgfmathsetmacro\osiconsheight{25}%

	\begin{tikzpicture}[x=1pt,y=1pt,>=Latex]
		\tikzset{comp/.style={
			draw,
			rectangle,
			fill=bg_color,
			minimum width=\computerwidth,
			minimum height=\computerheight
		}}
		\tikzset{module/.style={
			draw,
			rectangle,
			anchor=north west,
			minimum height=17,
			fill=color2,
			outer sep=0pt,
		}}
		\tikzset{dll/.style={
			draw,
			rectangle,
			anchor=west,
			minimum height=17,
			font=\footnotesize,
			fill=color3,
			text opacity=1,
		}}

		%---------------------------------
		% Computers
		\node[comp] (c1) at (0,0) {};
		\node[anchor=south] (c1t) at (c1.north) {Ordinateur 1};
		\node[anchor=south west] (win) at (c1.south west) {\includegraphics[height=\osiconsheight pt]{logos/Windows.png}};
		\node[anchor=west] (tux) at (win.east) {\includegraphics[height=\osiconsheight pt]{logos/Tux.png}};
		\node[anchor=west] (ect) at (tux.east) {\large\ldots};

		\node[comp] (c2) at (290,0) {};
		\node[anchor=south] (c2t) at (c2.north) {Ordinateur 2};
		\node[anchor=south west] (win) at (c2.south west) {\includegraphics[height=\osiconsheight pt]{logos/Windows.png}};

		%---------------------------------
		% Stub
		\node[
			isosceles triangle,
			isosceles triangle apex angle=44,
			xshift=90,
			yshift=40,
			minimum width=112.5,
			fill=cyan,
			fill opacity=0.20,
			blue,
			anchor=east,
			outer sep=0,
			inner sep=0
		] (watch) at (c1.center) {};

		\node[
			draw,
			diamond,
			aspect=2.5,
			fill=color1,
			text opacity=1,
			inner sep = 5,
			anchor=east,
			outer sep=0
		] (stub) at (watch.east) {Stub};

		%---------------------------------
		% Application
		\node[
			draw,
			circle,
			aspect=2.5,
			fill=green!40,
			xshift=-50,
			yshift=40,
			text opacity=1
		] (app) at (c1.center) {Application};

		%---------------------------------
		% GenDbg
		\node[
			draw,
			diamond,
			aspect=2.5,
			fill=color1,
			xshift=-55,
			yshift=40,
			text opacity=1
		] (gendbg) at (c2.center) {GenDbg};

		%---------------------------------
		% GenDbg <-> Stub
		\draw[
			<->,
			dotted,
			thick,
			label=above
		] (gendbg) to[bend right] node[above,pos=0.5] {COM} (stub);

		%---------------------------------
		% GUI
		\node[
			module,
			anchor=south west,
			xshift=\xstartmodules,
			yshift=-\ystartmodules,
		] (gui) at (gendbg.north east) {GUI};
		\node[dll] (dll) at (gui.east) {dll};

		%---------------------------------
		% AsmModule
		\node[
			module,
			xshift=\xstartmodules,
			yshift=\ystartmodules,
		] (asmmodule) at (gendbg.south east) {AsmModule};
		\node[dll] (dll) at (asmmodule.east) {dll};

		%---------------------------------
		% CmdModule
		\node[
			module,
			yshift=\yshiftmodules,
		] (cmdmodule) at (asmmodule.south west) {CmdModule};
		\node[dll] (dll) at (cmdmodule.east) {dll};

		%---------------------------------
		% SymModule
		\node[
			module,
			yshift=\yshiftmodules,
		] (symmodule) at (cmdmodule.south west) {SymModule};
		\node[dll] (dll) at (symmodule.east) {dll};

		%---------------------------------
		% OSHelperModule
		\node[
			module,
			yshift=\yshiftmodules,
		] (oshelpermodule) at (symmodule.south west) {OSHelperModule};
		\node[dll] (dll) at (oshelpermodule.east) {dll};

		%---------------------------------
		% BpModule
		\node[
			module,
			yshift=\yshiftmodules,
		] (bpmodule) at (oshelpermodule.south west) {BpModule};
		\node[dll] (dll) at (bpmodule.east) {dll};

		%---------------------------------
		% GenDbg -> modules
		\draw[-Square] (gendbg.north east) |- (gui);
		\draw[-Square] (gendbg.south east) |- (asmmodule);
		\draw[-Square] (gendbg.south east) |- (cmdmodule);
		\draw[-Square] (gendbg.south east) |- (symmodule);
		\draw[-Square] (gendbg.south east) |- (oshelpermodule);
		\draw[-Square] (gendbg.south east) |- (bpmodule);
	\end{tikzpicture}
\end{document}
