%!TEX root = ./RAPPORT_A17_INFO_ST40_PINARD_MAXIME.tex

%----------------------------------------
% Figures configuration
%----------------------------------------

\usetikzlibrary{shapes}
\usetikzlibrary{arrows.meta}
\usetikzlibrary{calc}

\definecolor{bg_color}{RGB}{250,250,229}

\colorlet{color1}{cyan!50}
\colorlet{color2}{red!30!green!40}
\colorlet{color3}{orange!50}
\colorlet{color4}{violet!60!blue!55}

\newganttlinktype{bartobardown}{
	\ganttsetstartanchor{south east}
	\ganttsetendanchor{north west}
	\draw [/pgfgantt/link] (\xLeft, \yUpper) -- (\xRight, \yLower);
}
\newganttlinktype{bartobarup}{
	\ganttsetstartanchor{north east}
	\ganttsetendanchor{south west}
	\draw [/pgfgantt/link] (\xLeft, \yUpper) -- (\xRight, \yLower);
}
\newganttlinktype{milestonetobardown}{
	\ganttsetstartanchor{south}
	\ganttsetendanchor{north west}
	\draw [/pgfgantt/link] (\xLeft, \yUpper) -- (\xRight, \yLower);
}
\newganttlinktype{bartomilestonedown}{
	\ganttsetstartanchor{south east}
	\ganttsetendanchor{north}
	\draw [/pgfgantt/link] (\xLeft, \yUpper) -- (\xRight, \yLower);
}


%----------------------------------------
% upmethodology commands redefinition
%----------------------------------------

\makeatletter

% Remove 'Initials' column from validators
\renewcommand{\upm@document@addvalidator}[3][]{%
	\global\protected@edef\thevalidatorlist{\thevalidatorlist\protect\Ifnotempty{\thevalidatorlist}{,} \protect\upmmakename{#2}{#3}{~}}

	\global\protected@edef\upm@document@validator@tab@commented{\upm@document@validator@tab@commented \protect\upmmakename{#2}{#3}{~} & 
	& \protect\Ifnotempty{#1}{\protect\href{mailto:#1}{#1}}\\}

	\ifupm@document@validator@tab@hascomment\else
		\global\protected@edef\upm@document@validator@tab{\upm@document@validator@tab \protect\upmmakename{#2}{#3}{~} & 
		\protect\Ifnotempty{#1}{\protect\href{mailto:#1}{#1}}\\}
	\fi
}
\renewcommand{\upm@document@addvalidatorstar}[4][]{%
	\global\protected@edef\thevalidatorlist{\thevalidatorlist\protect\Ifnotempty{\thevalidatorlist}{,} \protect\upmmakename{#2}{#3}{~}}

	\global\let\upm@document@validator@tab\relax

	\global\protected@edef\upm@document@validator@tab@commented{\upm@document@validator@tab@commented \protect\upmmakename{#2}{#3}{~} & 
	#4 & \protect\Ifnotempty{#1}{\protect\href{mailto:#1}{#1}}\\}

	\upm@document@validator@tab@hascommenttrue
}
\renewcommand{\upmdocumentvalidators}[1][\linewidth]{%
	\ifupm@document@validator@tab@hascomment%
		\Ifnotempty{\upm@document@validator@tab@commented}{%
		\noindent\expandafter\begin{mtabular}[#1]{3}{|X|l|c|}%
		\tabulartitle{\upm@lang@document@validators}%
		\tabularheader{\upm@lang@document@names}{\upm@lang@document@comments}{\upm@lang@document@emails}%
		\upm@document@validator@tab@commented
		\hline%
		\expandafter\end{mtabular}\par\vspace{.5cm}}%
	\else%
		\Ifnotempty{\upm@document@validator@tab}{%
		\noindent\expandafter\begin{mtabular}[#1]{2}{|X|c|}%
		\tabulartitle{\upm@lang@document@validators}%
		\tabularheader{\upm@lang@document@names}{\upm@lang@document@emails}%
		\upm@document@validator@tab
		\hline%
		\expandafter\end{mtabular}\par\vspace{.5cm}}%
	\fi%
}

% Remove history from document info page
\renewcommand{\upmdocinfopage}{
	\thispagestyle{plain}
	\upmdocumentsummary\upmdocumentauthors\upmdocumentvalidators\upmdocumentinformedpeople\clearpage%
}

% Decrease space after upmcaution upminfo and upmquestion message boxes
\renewenvironment{upm@highligh@box}[2]{%
	\par
	\vspace{.5cm}
	\begin{tabular}{|p{#1}|}
	\hline
	\begin{window}[0,l,{\mbox{\includegraphics[width=1cm]{#2}}},{}]
}{%
	\end{window}\\ \hline \end{tabular}
	%\vspace{.5cm}
	\par
}

\makeatother

%----------------------------------------
% upmethodology informations
%----------------------------------------

% Document Information and Declaration
\declaredocument{Rapport de stage ST40 - A2017}{Développement et fiabilisation d’un module de désassemblage interne}{-}

% Document Authors
\addauthor*[maxime.pinard@utbm.fr]{Maxime}{Pinard}{Étudiant en branche INFO}

% Document Validators
\addvalidator*[jean-charles.creput@utbm.fr]{Jean-Charles}{Creput}{Suiveur UTBM}
\addvalidator*[benoit.amiaux@intradef.gouv.fr]{Benoît}{Amiaux}{Tuteur en entreprise}

% Informed People
\addinformed*[jean-charles.creput@utbm.fr]{Jean-Charles}{Creput}{Suiveur UTBM}
\addinformed*[benoit.amiaux@intradef.gouv.fr]{Benoît}{Amiaux}{Tuteur en entreprise}
\addinformed*[laure.foissard@intradef.gouv.fr]{Laure}{Foissard}{Responsable administratif (DRH)}

% Copyright and Publication Information
\setcopyrighter{Maxime Pinard}
\setpublisher{l'Universitée de Technologie de Belfort Montbéliard}

% Version
\incversion{\makedate{\the\day}{\the\month}{\the\year}}{Version initiale.}{\upmpublic}

%----------------------------------------
% utbmcovers informations
%----------------------------------------

\UseExtension{utbmcovers}

\setutbmfrontillustration{cover}
\setutbmtitle{Développement et fiabilisation d’un module de désassemblage interne}
\setutbmsubtitle{Rapport de stage ST40 - A2017}
\setutbmstudent{Maxime Pinard}
\setutbmstudentdepartment{Département Informatique}
\setutbmstudentpathway{}
\setutbmcompany{Direction générale de l'armement\\Maîtrise de l’information}
\setutbmcompanyaddress{Route de Laillé\\ 35131 Bruz, France}
\setutbmcompanywebsite{\href{http://www.defense.gouv.fr/dga}{\color{utbm_cover_main_shadow_text}{http://www.defense.gouv.fr/dga}}}
\setutbmcompanytutor{Benoît Amiaux}
\setutbmschooltutor{Jean-Charles Creput}
\setutbmkeywords{
	% Branche d'activité économique
	Armement \textendash{}
	SSII, services informatiques \textendash{}
	% Métiers
	Informatique \textendash{}
	Etude, développement \textendash{}
	% Domaine technologique
	Génie logiciel \textendash{}
	Sécurité \textendash{}
	% Application physique directe
	Logiciel d'analyse de données \textendash{}
	Analyse de binaires
}
\setutbmabstract{
	TODO
}

%----------------------------------------
% Other configurations
%----------------------------------------

% Figures folder
\graphicspath{{figures/}}

% Figures counting
\counterwithout{figure}{chapter}

% Table counting
\counterwithout{table}{chapter}

% Source code formatting
\upmcodelang{cpp}

% Prevent page breaks in paragraphs
\predisplaypenalty=1000
\postdisplaypenalty=1000
\clubpenalty=1000

% Minimal space required in the bottom margin not to move the title on the next page
%\renewcommand{\bottomtitlespace}{.1\textheight}

% Links config, especialy for the table of contents
\hypersetup{
    colorlinks=true,
    linkcolor=black,
    urlcolor=blue,
    linktoc=all
}

% French language config
\frenchbsetup{StandardLayout=true,ReduceListSpacing=false,CompactItemize=false}

% Vertical alignement config
\raggedbottom{}

\setglossarystyle{list}

%----------------------------------------
% Functions definitions
%----------------------------------------

% Clear to the next left page
\newcommand*{\cleartoleftpage}{
  \clearpage \ifodd\value{page}\hbox{}\newpage\fi
}

% Paragraph with line break
\newcommand{\p}[1]{\paragraph{#1\\}}

% Function to print a warning sign
\newcommand{\dangersign}[1][2.5ex]
	{\renewcommand{\stacktype}{L}
		{\scaleto{\stackon[1pt]{\color{red}$\triangle$}{\fontsize{4pt}{4pt}\selectfont !}}{#1}}}

% Definition of \Witem for 'itemize' environment with a warning sign
\newcommand{\Witem}
{\item[\dangersign{}]}

\newcommand{\annexe}[1]{Annexe \ref{sec:#1}}
\newcommand{\file}[1]{\texttt{#1}}
\newcommand{\folder}[1]{\texttt{#1}}
\newcommand{\reg}[1]{\texttt{\$#1}}
